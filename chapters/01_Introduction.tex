\section{Introduction}\label{Introduction}


Networks today have evolved significantly, one of the most tangible examples of this evolution are Wireless Networks. The most common way to access Internet from home are home WLANs, usually referred as home WiFi. The variety of services and devices using the home WiFi to access Internet is vast. It is common today for a home user to stream a movie on his laptop while connected to the home WiFi. In the ideal case scenario the experience is enjoyable, video team plays smoothly. In many cases, when the movie streaming is degraded, the experience is frustrating. One of the potential causes of poor streaming experience is the home WiFi. In fact, previous works \cite{homeoraccesslink} have identified home WLANs as the bottleneck along the service path. The cause of poor Home WLAN experience can be varied, as described in previous work \cite{wislow}, channel congestion, poor client or AP placement and interference are the most common causes. Other works \cite{wifi_weakest_link} have analyzed the impact of Home WLAN on latency along a network path. They have identified that WiFi hops latency can contribute up to 60\% of the overall round trip time along the service path. On top of these technical causes, a business risk arises. A risk between home users, ISPs and content providers. As described in previous works \cite{predicting_effect_Home_Wifi} the frustration is not only experienced by the users but also by users' ISPs who are often held responsible for poor Internet experience. This problem might seem small, nevertheless it can escalate until a point in which content providers lose their subscribers. Home users, in the search of a solution can switch between ISPs, if complication persists they can even switch content providers. In this context ISPs and content providers have little to none impact on one of the most common root causes of the degraded experience, the home WLAN. Under this light lies our motivation to develop a tool to identify Wireless impairments in Home WLANs. The description of the initial stages of this tool are presented along this paper. Identifying where the root cause is within the Home WLAN is challenging due to multiple factors. To begin with, Wireless nature is unreliable as it uses an open and shared medium, shared among WiFi and non-WiFi devices. Another factor is to choose the most suitable measurement technique to find where the problem is. At the time of this paper and to the best of our knowledge, most research works have mainly implemented passive techniques \cite{hostview} \cite{passive_wifi_capacity_estimation}. A couple others have relied upon active techniques \cite{can_user_level_probing}. Depending on the type of measurement technique chosen challenges can be presented. Passive techniques face the challenge of requiring access to the device collecting the metrics. With active techniques the complication is tied to overhead caused by the measurement tool. In other words, with active techniques the very same measurement instrument can bias the measured metrics. Our tool implements a mixture of both to take strong points of both and leverage the weakness with each other's strong points. Further description of these techniques along with related work associated to home WLAN study will be covered in section \ref{Back_Related_Work}. The instrumentation details of our tool are developed in section \ref{Wireless Bottleneck Detector}. The mechanisms and techniques to evaluate our method to identify impairments in Home WLANs is explained in section \ref{Evaluation_Method}. Finally findings of our work are consolidated in section \ref{Results}.
 