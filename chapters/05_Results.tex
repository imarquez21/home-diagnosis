\section{Results}\label{Results}



In this initial stage the results obtained from random forest classification are promising nevertheless they can be considered ``too good to be true". We consider that it is required for us to include more variability to the experiments we are working with and the size of the datasets. Therefore, our next step is to increase the variability in our experiments and the size of the datasets to increase the robustness of our results. As mentioned in the opening of this paper, the goal is to include the findings of this initial stage to leverage the wide deployment of a home WiFi impairments detector in the wild. Our work is the spearhead of this tool, there is room for improvement and further development to be accomplished.

%These results summarize the first stage of the tool we strive to develop. We can conclude that the potential algorithm to be used for our tool to classify, and therefore detect home WiFi impairments, can be random forest. We consider it a potential algorithm as it achieved the best results from both datasets created from our experiments. We can also conclude that the set of features and metrics collected in our experiments result in an high accuracy classification level. The next step is to extend the number of samples we collected to test our results against a bigger dataset to increase the robustness of our results. The final goal is to deploy the WiFi impairment detector in the wild.

% By running these test we have validated that the amount of attribute defined helps us to properly classify the algorithms.
\newpage


