\section{Background and Related Work}\label{Back_Related_Work}

The challenge to identify issues in the Home WiFi has been approached before. To address this challenge the research community has relied on two measurement techniques, active and passive. While most of previous works have opted for passive techniques \cite{hostview} \cite{passive_wifi_capacity_estimation}, others, have worked with active ones \cite{can_user_level_probing}. The work of Da Hora et al \cite{passive_wifi_capacity_estimation} chose passive techniques. In their research context, active techniques might have led to user traffic disruption and battery drain of devices under study. Within the context of passive metrics, they excluded per packet analysis as it can result in overhead during high network utilization periods. Their work mostly relied on standard metrics passively collected from APs. Active techniques were implemented in the work of Kanuparthy et al \cite{can_user_level_probing}. Their work rely on user-level probing. They propose a metric called one-way-delay OWD or wireless access delay. The OWD reflects the delays a packet faces while going through a 802.11 link. They have chosen active measurement to achieve software and hardware agnosticism. They pursue agnostic mechanisms to facilitate the deployment of the tool at a large scale. The common ground among the works mentioned before and our work is tool's usability and scalability. We strive for a tool to be deployed at a large scale with minimal modifications to the Home WiFi setup. The area in which we differ with previous works is the implementation of active probing with a specific probing rate. The probing rate we have chosen will get a sense of network status without adding significant overhead to it. We describe how we chose the probing rate in Section \ref{Wireless Bottleneck Detector}.
