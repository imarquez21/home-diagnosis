\section{Related Work}\label{Back_Related_Work}

The challenge to identify issues in the Home WiFi has been approached before. To address this challenge the research community has relied on two measurement techniques, active and passive. While most of previous works have opted for passive techniques \cite{hostview}, \cite{passive_wifi_capacity_estimation}, \cite{observing_through_wifi_APs}; others, have worked with active ones \cite{can_user_level_probing}. The work of Da Hora et al \cite{passive_wifi_capacity_estimation} chose passive techniques. In their work they develop  a method to detect poor QoE derived from WiFi quality metrics. In their research context, active techniques have not been used to prevent user traffic disruption and battery drain of devices under study. Within the context of passive metrics, they excluded per packet analysis as it can result in overhead during high network utilization periods. Our efforts learn from their work on the meaningful WiFi quality metrics to be passively collected. On the active probing side, we approach it a different way as we implement an active probing rate which we believe will not cause user traffic disruption. In the work Ashish et al \cite{observing_through_wifi_APs} they present a metric called \emph{Witt} which can help to get a sense of the Home WiFi experience from the AP perspective. Their implementation relies on passive measurements. Their AP, which is a custom AP, requires to be customized in hardware and software to interact with their infrastructure. From their work we takeaway the strong point of using the AP as a vantage point to get a sense of the home WiFi experience. However, we discard AP customization even though it increases the granularity and number of metrics collected at the cost of deployability. Our work pursues a software and hardware agnostic tool to facilitate its wide deployment in the wild.
On the active measurements techniques side, Kanuparthy et al \cite{can_user_level_probing} have implemented user-level probing. In their work they describe the ability to identify wireless pathologies derived from a metric proposed by them called one-way-delay OWD or wireless access delay. The OWD metric reflects the delays a packet faces while going through a 802.11 link. Their setup is almost agnostic as they require to deploy a wired device in the home WiFi for their metric to be collected. Our works is similar to them as we also implement user-level probing, ping-like tools to achieve software and hardware agnosticism. We approach the limitation of having a wired device deployed in the home WiFi by leveraging with an existing project which facilitates the use of the wired device or server \cite{hostview}. In other words, even though our focus is on active measurement techniques, we leverage the use of tools to collect passive metric to deploy our active measurement tool. Under the same light of active measurement techniques, the work of Syrigos et al \cite{WLAN_Troubleshooting} defines a set of metrics to characterize WiFi pathologies. The metrics proposed are derived from statistic available in most wireless devices. From their work we learn active metrics available in wireless equipment from which WiFi impairments pathologies can be characterized.
For the development of this paper we have learned from research mention along this section and continue to extend our knowledge on novel frameworks to approach the challenge of Home WiFi impairment detection.

%They have chosen active measurement to achieve software and hardware agnosticism. They pursue agnostic mechanisms to facilitate the deployment of the tool at a large scale. The common ground among the works mentioned before and our work is tool's usability and scalability. We strive for a tool to be deployed at a large scale with minimal modifications to the Home WiFi setup. The area in which we differ with previous works is the implementation of active probing with a specific probing rate. The probing rate we have chosen will get a sense of network status without adding significant overhead to it. We describe how we chose the probing rate in Section \ref{Wireless Bottleneck Detector}.
