\begin{abstract}
The preferred method to access Internet from home is WiFi. Unfortunately poorly placed WiFi access point can experience WiFi impairments such as interference or congestion, leading to degraded Internet performance. Identifying these impairments can be challenging, even for wireless experts. To approach this challenge we develop a tool to identify home WiFi impairments. In our work we conduct experiments triggering Wireless and non-wireless issues in a testbed. The two methods we work with are active probing and wireless metrics collection from Wireless AP and Wireless Client. The wireless metrics we collect include but are not limited to, RSSI, PHY Rate, Noise, etc. With these metrics we get a sense of the home WiFi and correlate it with our active probing results. Finally, to identify a Wireless impairment we run our dataset through supervised learning algorithm. We obtain the best results with random forest algorithm. Random forest is well known for its precision to classify events based on specific feature. We close our paper by presenting the results of our impairment detection by modeling it through a classification task.
\end{abstract}