\section{Evaluation Method}\label{Evaluation_Method}

\begin{table*}[!bp]
	\begin{center}
		\begin{tabular}{|| m{2.5cm} | m{2cm} | m{2cm}| m{2cm} | m{2cm}| m{2cm} | m{2cm}  ||}
			\hline
			%\diagbox{Attribute}{Algorithm}
			Attribute-Algorithm & Dataset & Correctly Classified & Incorrectly Classified & RMSE & Precision & Recall \\ [0.5ex] 
			\hline\hline
			\multirow{2}{*}{J.48} & Binary & 99.47\% & 0.53\% & 0.06 & 0.99 & 0.99 \\
			& Multiclass & 98.66\% & 1.33\% & 0.07 & 0.98 & 0.98 \\
			\multirow{2}{*}{Adaboost} & Binary & 99.73\% & 0.27\% & 0.05 & 0.99 & 0.99 \\
			& Multiclass & 85.73\% & 14.26\% & 0.20 & 0.99 & 0.85 \\
			\multirow{2}{*}{Random Forest} & Binary & 100.00\% & 0.00\% & 0.03 & 1.00 & 1.00 \\
			& Multiclass & 99.86\% & 0.13\% & 0.04 & 0.99 & 0.99 \\
			\hline
		\end{tabular}
		\caption{Classification Results}
		\label{table:classification_results}
	\end{center}
\end{table*}

To evaluate the classification generated by Weka we create two datasets, binary and multiclass. We create these datasets to validate the precision of the Weka algorithms implementation. In the binary dataset we work with only two labels. The goal is to identify if there is a WiFi impairment or not. The labeling schema for the binary dataset is described in table \ref{table:binary_labels}.

\begin{table}[H]
	\begin{center}
		\begin{tabular}{||c c||} 
			\hline
			Label & Issue Type\\ [0.5ex] 
			\hline\hline
			0 & No Issue at all \\ 
			\hline
			1 & Attenuation\\
			\hline
			0 & Access-Link Limiting \\
			\hline
			1 & Congestion \\[1ex] 
			\hline
		\end{tabular}
		\caption{Binary dataset labels}
		\label{table:binary_labels}
	\end{center}
\end{table}

As represented in table \ref{table:binary_labels}, even though ``access link limiting" is a network impairment it shares the same label as ``no issue at all". They both share the same label as access link limiting nature is non-Wireless.
The second type of dataset is multiclass. The multiclass dataset has the goal to classify in more detail the impairment, to distinguish if it is congestion, access link limiting, attenuation or none of them. In multiclass dataset we assign a different label to each impairment. The labeling for the multiclass dataset is described in table \ref{table:multiclass_labels}.

\begin{table}[H]
	\begin{center}
		\begin{tabular}{||c c||} 
			\hline
			Label & Issue Type\\ [0.5ex] 
			\hline\hline
			0 & No Issue at all \\ 
			\hline
			1 & Attenuation\\
			\hline
			2 & Access-Link Limiting \\
			\hline
			3 & Congestion \\[1ex] 
			\hline
		\end{tabular}
		\caption{Multiclass dataset labels}
		\label{table:multiclass_labels}
	\end{center}
\end{table}

For both datasets we ran the algorithms J.48, AdaBoostA1 and random forest. We used the default 10 fold cross-validation in Weka for these algorithms. The results obtained from the binary dataset are outlined in table \ref{table:classification_results}. The best results are obtained with the ``random forest" algorithm. The next step was to feed Weka with the multiclass dataset. As mentioned before, with the multiclass dataset the goal is to classify the impairment in more detail. The results are summarized in table \ref{table:classification_results}. Once again the best results are obtained with the random forest algorithm.



