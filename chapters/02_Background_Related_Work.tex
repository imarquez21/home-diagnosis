\section{Background and Related Work}\label{Back_Related_Work}

The challenge to identify issues in the Home WiFi has been approached before. To address this challenge the research community has relied on two measurement techniques, active and passive. While most of previous works have opted for passive techniques \cite{hostview} \cite{passive_wifi_capacity_estimation}, others, have worked with active ones \cite{can_user_level_probing}. In the work of Joumblatt, Diana, et al. \cite{hostview} they have opted for passive techniques to be able to extract fine-grained data from packet captures. In their experiments context active techniques were not chosen as they might not reflect application performance properly. Application performance was a key element of their research. They overcame privacy concerns associated by anomymizing the collected data. The work of Neves Da Hora, et al \cite{passive_wifi_capacity_estimation} also chose passive techniques. In their research context, active techniques might have led to user traffic disruption and battery drain of devices under study. Within the context of passive metrics, they excluded per packet analysis as it can result in overhead during high network utilization periods. Their work mostly relied on standard metrics passively collected from APs.
Active techniques were implemented in the work of Kanuparthy, Partha, et al \cite{can_user_level_probing}. Their work rely on user-level probing. They propose a metric called one-way-delay OWD or wireless access delay. The OWD reflects the delays a packet faces while going through a 802.11 link. They have chosen active measurement to achieve software and hardware agnostic mechanisms. They pursue agnostic mechanisms to facilitate the deployment of the tool at a large scale. The common ground among the works mentioned before and our work is tool's usability and scalability. We strive for a tool to be deployed at a large scale with minimal modifications to the Home WiFi setup. The area in which we differ with previous works is the implementation of an active measurement technique with a specific probing rate. The probing rate we have chosen will get a sense of network status without adding significant overhead to it. We describe how we chose the probing rate in Section \ref{Wireless Bottleneck Detector}.


\subsection{Wireless Monitoring Metrics}\label{Wireless Monitoring Metrics}

 Active and passive techniques have their own strong points and areas for improvement. In the following lines we outline the main characteristics of each one of them and what can be considered their strengths and weaknesses. Important to mention, we do not dare to tell a specific technique is better than the other. Each of the techniques will be best-suited depending on the  goal and context of the experiment.

\subsection*{Active}

Active measurement techniques are mainly characterized by its ability to capture the state of the network in almost real-time. In other words, active measurement can help to identify a condition when is present in the network. This characteristic is different from passive measurements which can be considered historical. Active measurements are also characterized by the use of probes. Probes are packets ``injected" in the network to measure its status. For example, ping relies on ICMP requests and replies to compute the Round-Trip Time. For Ping, the probes are the IMCP requests and replies. It is important to pay attention to the probe size and probing rate. Probes can add overhead to the network if their size is large compared to the capacity of the path or if the rate is high. If probing causes overhead it will not only might disrupt user traffic but can also lead to biased measurement results. In the following bullet points we outline the strengths and weaknesses of active measurement techniques.

\textbf{Strengths}
\begin{itemize}
	\item Full ownership of the network is not required.
	\item They do not require large space to store data collected as generally, probe packets are small.
	\item Privacy concerns are minimal as probe packet used to measure are made of random data which has no sensitive information.
	\item Useful to get the state of the network in almost real-time.
\end{itemize}
	

\textbf{Weaknesses}
\begin{itemize}
	\item Overhead might occur if probe size and rate are chosen without due diligence of network conditions.
	\item Biased results can be obtained if probing causes overhead in the network.
	\item They can only capture an instant of the network condition. If problem to be characterized is extended in time, active measurement might not measure it accurately.
\end{itemize}


Under the scope of active measurement techniques, the following are the metrics to be actively collected for our work.

\subsection*{Active Metrics}

\begin{itemize}
	%\item One-Way Delay
	\item \textbf{Round Trip Time}
	\begin{itemize}
		\item This metric takes into account the time it takes for a probe to leave the source, reach the destination and come back to the source. In our work we will compute statistics from RTT.
	\end{itemize}
	
	\item \textbf{Throughput}
	\begin{itemize}
		\item The amount of data sent or received from or by a station within a time window. 
	\end{itemize}
	%\item \textbf{Losses}
	%\begin{itemize}
	%	\item ICMP reply packets not coming back to the source.
	%\end{itemize}
\end{itemize}

\subsection*{Passive}

Passive measurement techniques rely on a ``listen and sit" approach. The instrument conducting passive measurements in the network sits in a specific location along the path and records the metrics of interest. The instrument can be a component of the network itself, for example a router. It can also be device devoted to measure, such as a Wireless sniffer. An important difference between active and passive techniques, is that passive tend to be historical whereas active are real-time oriented. In an historical sense, passive measurements are more reliable to characterize a network problem which covers an extended time-frame. Active measurements are best-suited to pinpoint a problem in the instant it happens, nevertheless they lack accuracy to characterize problems covering an extended time-frame. Another difference between active and passive measurements is that the latter do not trigger probes. Overhead due to probe packets is not present in passive measurements. However, computational and storage resources in the passive measuring device are important factors to consider. The device might require to have enough space to store the data being collected. In a similar way, the computational power of the device can be required to be high depending on the speed of the link being measured. A Gigabit link in a Core Router will handle significantly more data than an 100Mbps Ethernet link in an access switch. 
Outlined in the following list a high level summary of the strengths and weaknesses of passive measurement techniques.

\textbf{Strengths}
\begin{itemize}
	\item No extra traffic is generated to collect metrics, risk of causing overhead is minimized.
	\item They are best-suited to accurately characterize network problems covering an extended time frame.
	\item In general, they are able to collect large datasets leading to fine-grained data. Ultimately leading to increased network complications diagnosis accuracy.

\end{itemize}


\textbf{Weaknesses}
\begin{itemize}
	\item Large storage capacity can be required to store collected data. Not all measuring devices have large storage capacity, i.e. Access Points.
	\item Access to equipment working as passive measurement device is required. This is not possible for most users at multiple devices along an Internet path.
	\item High computational power on the measuring device can be required depending on the link being monitored and data granularity pursued. Not all devices can provide high computational power, i.e. Access Points.
	\item They are reactive, findings on the network problem can be obtained after collected data has been analyzed. The majority of wireless interference issues are known to be short, 1 - 7 min \cite{observing_through_wifi_APs}. By the time the data has been analyzed the wireless interference issue might be already over.
\end{itemize}

\subsection*{Passive Metrics}

\begin{itemize}
	%\item \textbf{PHY Tx Rate}
	%\begin{itemize}
	%	\item The speed at which the device is connected to. Bitrate adaptation techniques are triggered based on channel conditions. Therefore this metric can assist to estimate the channel conditions.
	%\end{itemize}
	\item \textbf{RSSI - Received Signal Strength Indicator}
	\begin{itemize}
		\item The power at which the signal is being received by the device. Depending on the type of traffic, specific RSSI thresholds are often defined to set an acceptable RSSI level. For example, for VoIP the min RSSI value for an acceptable VoIP call is -67 dBm \cite{Cisco_VoWLAN_Guide}.
	\end{itemize}
	%\hfill \break
	%\item \textbf{Busy Time}
	%\begin{itemize}
		%\item This metric is associated to the time the channel was busy, in other words the time channel was being used and therefore not eligible for data exchange. Reasons for channel busy time to be close to 100\% can be contention by other Wi-Fi devices or interference from non-Wi-Fi sources.
	%\end{itemize}
	
	\item \textbf{PHY Tx Rate}
	\begin{itemize}
		\item 	The rate at which without medium access control, error correction or scheduling events the device is expected to operate with.
	\end{itemize}
	
	\item \textbf{Noise}
	\begin{itemize}
		\item  The noise perceived in the Wireless environment, high noise levels can degrade Wireless link quality.
	\end{itemize}
	
	\item \textbf{Throughput - Driver Logs}
	\begin{itemize}
		\item 	For this metric, we extract the throughput perceived by the Wireless driver debug logs. It depicts the effective amount of data that can be exchanged.
	\end{itemize}
	
	\item \textbf{Frame Delivery Ratio}
	\begin{itemize}
		\item Frame Delivery Ratio depicts the ratio between packets successfully received and total packet sent. The FDR metric can assist to get a sense of link quality. If FDR ratio is high then, the quality of the link can be perceived as good.
	\end{itemize}

\end{itemize}

\subsection{Where do we collect them?}

In our work we have collected the metrics described in section \ref{Wireless Monitoring Metrics}, from multiple devices, vantage points. The device from where we have collected the metrics is key to identify which device is perceiving a particular Home WiFi issue. The metrics values can differ depending on the vantage point, the difference can be caused by device placement, OS, resources, driver and many more.

\begin{itemize}
	\item \textbf{Throughput - Active} - We have actively measured throughput from the wired client. As the goal is to test the link between Wireless client towards the Wired Client and going through the AP, we have collected the data at the wired client. Previous works have identified that even with similar Wireless conditions devices can experience different throughput and bitrates \cite{measuring_user_traffic}. We use iPerf as the tool to collect this metric. Further details on iPerf setup to collect this metric are described in section \ref{Wireless Bottleneck Detector}.
	
	\item \textbf{PHY Tx Rate} - We have extracted this metric from WiFi logs at AP and Wireless client. The goal is to identify at which rate was the last frame prior to collecting the log sent. We have also added a wireless sniffer to collect packet captures of wireless traffic between wireless client and AP.
	
	\item \textbf{RSSI} - We extract logs from the Wireless client and the AP to obtain RSSI data from each of them. Different logs have been collected at these two vantage points to validate its accuracy.
	
	\item \textbf{Noise} - A factor contributing to Wireless degradation is Noise, it is the Wireless interference coming from non-Wi-Fi sources. This can be caused by Microwave ovens, cordless phones and similar devices which "do not speak Wi-Fi language". Noise will be measured at both ends, wireless client and AP. We strive to identify which one experiences higher noise levels to pinpoint where the complication might be located.
	
	\item \textbf{Frame Delivery Ratio - FDR } - To compute the FDR we have fetched driver debug logs from the AP and the Wireless client. The FDR can assist to identify which component, AP or Wireless device is experiencing a poor Wireless condition. For example when experiencing congestion, the AP can have a lower FDR than the AP as the AP location do not experience high channel utilization.
	
	\item \textbf{Throughput - Driver Logs} - From driver debug logs we collected Throughput at the AP and Wireless client. The goal is to validate the closeness between the perceived passive throughput between AP and wireless client.
	
	\item \textbf{RTT} - At the wired client we issue pings towards the wireless client. We log the ping output at the wired client to compute statistics from the RTT. Statistics collected from the RTT are minimum, average, maximum, standard deviation and losses.
	
	%\item \textbf{Losses} - From the Wired client we issue a pings toward the Wireless client. At the Wired client we log the output from the ping session and compute statistics from it. Loss rate is one of the statistics collected.
	
	%\item \textbf{Busy Time} - We will measure it from the AP as it is the one servicing clients in a specific channel (channels), meaning that clients will be connecting to that serving channel. If the busy time of the channel is high, means there is interference or other AP and clients using the same channel. A corrective action can be switching to a less busy channel. The busy time is made of two components, WiFi (congestion) and non-Wi-Fi (interference). Command \emph{iw INTERFACE survey dump}	
		
\end{itemize}
		%\item Server
