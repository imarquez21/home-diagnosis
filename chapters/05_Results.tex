\section{Results}\label{Results}

\begin{table}[bp]
	\begin{center}
		\begin{tabular}{|| m{2.5cm} | m{2cm} | m{2cm}| m{2cm} | m{2cm}| m{2cm} | m{2cm}  ||}
			\hline
			%\diagbox{Attribute}{Algorithm}
			 Attribute-Algorithm & Dataset & Correctly Classified & Incorrectly Classified & RMSE & Precision & Recall \\ [0.5ex] 
			\hline\hline
			\multirow{2}{*}{J.48} & Binary & 99.47\% & 0.53\% & 0.06 & 0.99 & 0.99 \\
			& Multiclass & 98.66\% & 1.33\% & 0.07 & 0.98 & 0.98 \\
			\multirow{2}{*}{Adaboost} & Binary & 99.73\% & 0.27\% & 0.05 & 0.99 & 0.99 \\
			& Multiclass & 85.73\% & 14.26\% & 0.20 & 0.99 & 0.85 \\
			\multirow{2}{*}{Random Forest} & Binary & 100.00\% & 0.00\% & 0.03 & 1.00 & 1.00 \\
			& Multiclass & 99.86\% & 0.13\% & 0.04 & 0.99 & 0.99 \\
			\hline
		\end{tabular}
		\caption{Classification Results}
		\label{table:classification_results}
	\end{center}
\end{table}

After completing our experiments we moved on to the classification phase. We approach of detector by classifying using supervised learning algorithms to classify the events in our experiments. Our experiments are designed in a way in which each impairment happens during a specific interval within the 10 min experiment window. In other words, each experiment session consists in 10 intervals, each one 60 sec long. Depending on the scenario; congestion, attenuation or access link, we know at which interval is the issue happening. We label each interval depending on the impairment. From our results we create two datasets, binary and multiclass. In the binary dataset we work with only two labels. The labeling schema for the binary dataset is described in table \ref{table:binary_labels}.

\begin{table}[!h]
	\begin{center}
		\begin{tabular}{||c c||} 
			\hline
			Label & Issue Type\\ [0.5ex] 
			\hline\hline
			0 & No Issue at all \\ 
			\hline
			1 & Attenuation\\
			\hline
			0 & Access-Link Limiting \\
			\hline
			1 & Congestion \\[1ex] 
			\hline
		\end{tabular}
		\caption{Binary dataset labels}
		\label{table:binary_labels}
	\end{center}
\end{table}

As represented in table \ref{table:binary_labels}, even though ``access link limiting" is a network impairment it shares the same label as ``no issue at all". They both share the same label as access link limiting nature is non-Wireless.
The second type of dataset is multiclass. In multiclass dataset we assign a different label to each impairment. The labeling for the multiclass dataset is described in table \ref{table:multiclass_labels}.

\begin{table}[!h]
	\begin{center}
		\begin{tabular}{||c c||} 
			\hline
			Label & Issue Type\\ [0.5ex] 
			\hline\hline
			0 & No Issue at all \\ 
			\hline
			1 & Attenuation\\
			\hline
			2 & Access-Link Limiting \\
			\hline
			3 & Congestion \\[1ex] 
			\hline
		\end{tabular}
		\caption{Multiclass dataset labels}
		\label{table:multiclass_labels}
	\end{center}
\end{table}

To test classification in our dataset we use \emph{Weka}. Weka is a software which has different supervised learning algorithms. We focus on the classification feature and feed Weka with our datasets to test the accuracy of different algorithms. For both datasets we ran the algorithms J.48, AdaBoostA1 and random forest. We used the default 10 fold cross-validation in Weka for these algorithms. The results obtained from the binary dataset are outlined in table \ref{table:classification_results}. The best results are obtained with the ``random forest" algorithm. The next step was to feed Weka with the multiclass dataset. As mentioned before, with the multiclass dataset the goal is to classify the impairment in more detail. The results are summarized in table \ref{table:classification_results}. Once again the best results are obtained with the random forest algorithm. 

In this initial stage the results obtained from random forest classification are promising nevertheless they can be considered "too good to be true". We consider that it is required for us to include more variability to the experiments we are working with and the size of the datasets. Therefore, our next step is to increase the variability in our experiments and the size of the datasets to increase the robustness of our results. As mentioned in the opening of this paper, the goal is to include the findings of this initial stage to leveraged the development of a home WiFi impairments detector to be widely deployed in the wild. Our work is the spearhead of this tool, there is room for improvement and further development to be accomplished.

%These results summarize the first stage of the tool we strive to develop. We can conclude that the potential algorithm to be used for our tool to classify, and therefore detect home WiFi impairments, can be random forest. We consider it a potential algorithm as it achieved the best results from both datasets created from our experiments. We can also conclude that the set of features and metrics collected in our experiments result in an high accuracy classification level. The next step is to extend the number of samples we collected to test our results against a bigger dataset to increase the robustness of our results. The final goal is to deploy the WiFi impairment detector in the wild.

% By running these test we have validated that the amount of attribute defined helps us to properly classify the algorithms.
\newpage


