\section{Introduction}\label{Introduction}

The most common way to access Internet from home is Wi-Fi. The variety of services and devices using the home Wi-Fi to access Internet is vast. It is common today for a home user to stream a movie on his laptop while connected to the home Wi-Fi. In some cases, the movie streaming is degraded, which is frustrating for the users. One of the potential causes of poor streaming experience is the home Wi-Fi. In fact, previous work \cite{homeoraccesslink} has identified home Wi-Fi as the bottleneck along the end-to-end path. The cause of poor home Wi-Fi experience can be varied \cite{wislow}, channel congestion, poor client or AP placement and interference are the most common causes. Other work \cite{wifi_weakest_link} has analyzed the impact of Home Wi-Fi on the latency in a network path. They have identified that Wi-Fi latency can contribute up to 60\% of the overall round trip time along the end-to-end path. ISPs are often held responsible for poor Internet experience \cite{predicting_effect_Home_Wifi}. Home users, in the search for a solution can switch between ISPs or content providers even though the problem is within the home. In this research paper we develop a tool to identify home Wi-Fi impairments. We describe the initial stages of this tool in this paper. 

Identifying where the root cause is within the home Wi-Fi is challenging due to multiple factors. First, wireless nature is volatile as it uses an open and shared medium, shared among Wi-Fi and non-Wi-Fi devices. Second, it is required to have a vantage point within the home. This vantage point should be common across home deployments and capable of collecting Wi-Fi metrics to assist on the identification of Wi-Fi impairments. Most research work has mainly implemented passive techniques to identify where the potential cause for a degraded service is located \cite{hostview}, \cite{passive_wifi_capacity_estimation}, \cite{observing_through_wifi_APs}. A couple others have relied upon active techniques \cite{can_user_level_probing}, \cite{WLAN_Troubleshooting}. Depending on the type of measurement technique, it is required to address different considerations. Passive techniques face the challenge of requiring access to the AP to collect the metrics. Making changes to the AP to collect metrics is another challenge as most APs are not open to be customized. With active techniques the complication is the potential overhead caused by the measurement tool. In other words, with active techniques the network can experience disruption. 

Our tool implements both techniques to take strong points of both and leverage the weakness with each other's strong points. In this initial phase we are using both techniques to identify the relationship  between metrics passively collected and active probing results. We believe that mainly relying on active probing to identify home Wi-Fi impairments can be a breakthrough in the development of tools to be widely deployed in the wild. Further description of these techniques is covered in Section \ref{Wireless Monitoring Metrics}. Related work associated to home Wi-Fi study will be covered in Section \ref{Back_Related_Work}. The instrumentation details of our tool are developed in Section \ref{Wireless Bottleneck Detector}. The mechanisms and techniques to evaluate the method used to identify home Wi-Fi impairments is explained in Section \ref{Evaluation_Method}. Finally findings of our work are consolidated in Section \ref{Results}.
\newpage