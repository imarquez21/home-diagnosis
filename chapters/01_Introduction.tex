\section{Introduction}\label{Introduction}

The most common way to access Internet from home is WiFi. The variety of services and devices using the home WiFi to access Internet is vast. It is common today for a home user to stream a movie on his laptop while connected to the home WiFi. In some cases, the movie streaming is degraded, which is frustrating for the users. One of the potential causes of poor streaming experience is the home WiFi. In fact, previous work \cite{homeoraccesslink} has identified home WiFi as the bottleneck along the end-to-end path. The cause of poor home WiFi experience can be varied \cite{wislow}, channel congestion, poor client or AP placement and interference are the most common causes. Other work \cite{wifi_weakest_link} has analyzed the impact of Home WiFi on the latency at a network path. They have identified that WiFi latency can contribute up to 60\% of the overall round trip time along the end-to-end path. ISPs are often held responsible for poor Internet experience \cite{predicting_effect_Home_Wifi}. Home users, in the search for a solution can switch between ISPs or even content providers even though the problem is within the home. In this research paper we develop a tool to identify home WiFi impairments. We describe the initial stages of this tool this paper. 

Identifying where the root cause is within the Home WiFi is challenging due to multiple factors. First, wireless nature is volatile as it uses an open and shared medium, shared among WiFi and non-WiFi devices. Second, it is required to have a vantage point within the home. This vantage point should be common across home deployments and capable of collecting the measurement to identify an WiFi impairment. Most research work has mainly implemented passive techniques to identify where the potential cause for a degraded service is located \cite{hostview} \cite{passive_wifi_capacity_estimation}. A couple others have relied upon active techniques \cite{can_user_level_probing}. Depending on the type of measurement technique chosen challenges are present. Passive techniques face the challenge of requiring access to the AP to collect the metrics. Making changes to the AP to collect metrics is another challenge as most AP are not open to be customized. With active techniques the complication is the potential overhead caused by the measurement tool. In other words, with active techniques the measurements can bias the results. 

Our tool implements both techniques to take strong points of both and leverage the weakness with each other's strong points. In other words, in this initial phase we are using both techniques to see draw a pattern from metrics collected from the AP and how these correlate with the active probing results. We believe that mainly relying on active probing to identify home WiFi impairments can be a breakthrough in the development of tools to be widely deployed in the wild. Further description of these techniques along with related work associated to home WiFi study will be covered in Section \ref{Back_Related_Work}. The instrumentation details of our tool are developed in Section \ref{Wireless Bottleneck Detector}. The mechanisms and techniques to evaluate our method to identify impairments in Home WiFi are explained in Section \ref{Evaluation_Method}. Finally findings of our work are consolidated in Section \ref{Results}.
\newpage