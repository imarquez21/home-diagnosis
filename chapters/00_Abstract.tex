\begin{abstract}
Home WLANs have become an essential element in households nowadays. The preferred method to access Internet from home is WiFi. Home WLANs have brought their benefits and challenges into the home. The variety and complexity of WiFi and non-WiFi devices make Home WLANs keen to experience WiFi impairments. Identifying these impairments can be challenging, even for Wireless experts. To approach this challenge we have begun to develop a tool to identify WiFi issues in Home WLANs. In this paper we present the initial stages to develop the basis of this tool. We have conducted experiments triggering Wireless and non-wireless issues in a testbed. During these experiment sessions we have collected metrics from different components in the setup. Metrics have been collected using active and passive measurement techniques, a description of these two techniques is covered in section \ref{Wireless Monitoring Metrics}. Finally we consolidate and correlate these metrics to identify when a WiFi issue is happening.
\end{abstract}