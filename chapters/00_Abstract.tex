\begin{abstract}
The preferred method to access Internet from home is Wi-Fi. Unfortunately poorly placed Wi-Fi access point can experience Wi-Fi impairments such as interference or congestion, leading to degraded Internet performance. Identifying these impairments can be challenging, even for wireless experts. To approach this challenge we develop a tool to identify home Wi-Fi impairments. In our work we conduct experiments triggering wireless and non-wireless issues in a testbed. The two methods we work with are active probing and passive wireless metrics collection from wireless AP and wireless client. The wireless metrics we collect include but are not limited to, RSSI, PHY Rate, Noise, etc. With these metrics we get a sense of the status of home Wi-Fi and correlate it with our active probing results. Finally, to identify a wireless impairment we run our dataset through supervised learning algorithms. We obtain the best results with random forest algorithm. Random forest is well known for its precision to classify events based on a specific set of features. We close our paper by presenting the results of home Wi-Fi impairment detection by modeling it as a classification problem.
\end{abstract}