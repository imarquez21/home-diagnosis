\section{Wireless Monitoring Metrics}\label{Wireless Monitoring Metrics}

 Active and passive techniques have their advantages and disadvantages. In the following, we outline the main characteristics of each one of them. Each of the techniques will be best-suited depending on the  goal and context of the experiment.

\subsection{Active}

Active measurement is a technique in which traffic is injected in the network to get a sense of the network status. The injected packets are called probes. For our work we use active measurements to obtain metrics on bandwidth, Round-Trip Time (RTT) and packet loss. In the WiFi context, bandwidth active measurements can help to identify where the bottleneck is happening. We have also used active bandwidth measurement tools to generate traffic in our experimental setup to resemble real-case scenarios. High RTT can help to identify if congestion is happening in the home WiFi. In a similar way, packet loss can denote interference as frames are destroyed in the WiFi link. While using active measurements it is  important to pay attention to the probes size and probing rate. Large probe sizes and aggressive probing rate can cause overhead. Overhead does not only disrupts user traffic but can also lead to biased measurements. In the following bullet points we outline the strengths and weaknesses of active measurement techniques within our framework. We also include the ones we work with for this paper.


\textbf{Strengths}
\begin{itemize}
	\item Full ownership of the network is not required.
	\item They do not require large space to store data collected as generally, probe packets are small.
	\item Privacy concerns are minimal as probe packet used to measure are made of random data which has no sensitive information.
	\item Useful to get the state of the network on-demand.
\end{itemize}
	

\textbf{Weaknesses}
\begin{itemize}
	\item Overhead might occur if probe size and rate are chosen without due diligence of network conditions.
	\item Biased results can be obtained if probing causes overhead in the network.
\end{itemize}


Under the scope of active measurement techniques, the following are the metrics to be actively collected for our work.

\begin{itemize}
	%\item One-Way Delay
	\item \textbf{Round Trip Time}
	\begin{itemize}
		\item For our goal, RTT can helps us identify if we are experiencing attenuation and interference in the home WiFi. High RTT values can give a sense of latency in the Home WiFi which is potentially correlated to attenuation. Packet loss in the other hand, will point to interference related impairments as frames are being destroyed, causing the loss of these frames.
	\end{itemize}
	
	\item \textbf{Throughput}
	\begin{itemize}
		\item In WiFi, throughput active measurement can assist to identify if a congestion is happening in the WiFi link. For example, if the AP reports a strong signal to the wireless client and minimal losses but the throughput is low, it is likely that the AP is experiencing congestion.
	\end{itemize}
	%\item \textbf{Losses}
	%\begin{itemize}
	%	\item ICMP reply packets not coming back to the source.
	%\end{itemize}
\end{itemize}

\subsection{Passive}

Passive measurement techniques rely on a ``listen and sit" approach. The instrument conducting passive measurements in the network sits in a specific location along the path and records the metrics of interest. The monitor can be a component of the network itself, for example a router. It can also be device devoted to measure, such as a Wireless sniffer. An important difference between active and passive techniques is that the latter do not trigger probes. Overhead due to probe packets is not present in passive measurements. However, computational and storage resources in the passive measuring device are important factors to consider. The device might require to have enough space to store the data being collected. In a similar way, the computational power of the device can be required to be high depending on the speed of the link being measured. A Gigabit link in a Core Router will handle significantly more data than an 100Mbps Ethernet link in an access switch. 
Outlined in the following list a high level summary of the key strength and weaknesses of passive measurement techniques for our work purposes. We also outline the ones we use for our work.

\textbf{Strength}
\begin{itemize}
	\item No extra traffic is generated to collect metrics, risk of causing overhead is minimized.
\end{itemize}


\textbf{Weaknesses}
\begin{itemize}
	\item Large storage capacity can be required to store collected data. Not all measuring devices have large storage capacity, i.e. Access Points.
	\item Access to equipment working as passive measurement device is required. This is not possible for most users at multiple devices along an Internet path.
	\item High computational power on the measuring device can be required depending on the link being monitored and data granularity pursued. Not all devices can provide high computational power, i.e. Access Points.
\end{itemize}

\begin{itemize}
	%\item \textbf{PHY Tx Rate}
	%\begin{itemize}
	%	\item The speed at which the device is connected to. Bitrate adaptation techniques are triggered based on channel conditions. Therefore this metric can assist to estimate the channel conditions.
	%\end{itemize}
	\item \textbf{RSSI - Received Signal Strength Indicator}
	\begin{itemize}
		\item In our experiments we collect the RSSI from the AP and the Wireless Client. The RSSI help us to identify if there is attenuation happening in the link. A low RSSI denotes attenuation in the Wireless link.
		\item Low RSSI can be caused by poor AP placement due to large distance between wireless client and AP or obstacles between both.
	\end{itemize}
	%\hfill \break
	%\item \textbf{Busy Time}
	%\begin{itemize}
		%\item This metric is associated to the time the channel was busy, in other words the time channel was being used and therefore not eligible for data exchange. Reasons for channel busy time to be close to 100\% can be contention by other Wi-Fi devices or interference from non-Wi-Fi sources.
	%\end{itemize}
	
	\item \textbf{PHY Tx Rate}
	\begin{itemize}
		\item The PHY Tx Rate at the wireless nodes can illustrate poor WiFi link quality. A low PHY Tx Rate can help to diagnose a congestion, attenuation or interference WiFi impairment. In collaboration with other metrics the scope of the impairment can be narrowed down.
		\item For example, if the RSSI is strong, meaning there is no attenuation; loss rate is minimal, meaning interference is not present, but Tx PHY Rate is low; the impairment scope can be narrowed down to congestion.
	\end{itemize}
	
	\item \textbf{Noise}
	\begin{itemize}
		\item  Noise measurements assist know if environment where the wireless client or the AP is placed is suitable for WiFi. For example, if the noise level at a particular wireless client is high, we expect that node to be the only one with WiFi degraded quality. In the other hand, if the AP is the one sensing high noise levels, we can expect all the clients connected to that AP to experience degraded WiFi.
		\item Noise can be caused by devices which ``do not speak Wi-Fi language" such as Microwave ovens, cordless phones and similar. Noise can help to distinguish between congestion and interference. Unlike congestion, interference is driven by non-WiFi sources.
	\end{itemize}
	
	\item \textbf{Throughput - Driver Logs}
	\begin{itemize}
		\item The throughput from the driver logs assist us to sense a WiFi issue. Low throughput can be an indicator of congestion, attenuation or interference.
		\item In a similar way as for actively measuring throughput, in collaboration with other metric we narrow down the potential WiFi impairment.
		\item Additionally, passively measuring bandwidth help to validate we obtain similar values as for actively measuring it.
	\end{itemize}
	\newpage
	\item \textbf{Frame Delivery Ratio}
	\begin{itemize}
		\item Frame Delivery Ratio depicts the ratio between packets successfully received and total packet sent. The FDR metric can assist to get a sense of link quality. Low FDR indicates poor link quality. 
		\item Poor link quality can be caused due to congestion, attenuation or interference. In a similar way as with other metrics in our work, we collaborate with other metrics to narrow down the potential WiFi impairment being experienced.
	\end{itemize}

\end{itemize}

\subsection{Vantage Points}

The metrics described before have been collected from different devices in our setup. Previous works have identified that even with similar Wireless conditions devices can experience different throughput and bitrates \cite{measuring_user_traffic}, therefore we use different vantage points. Passive metrics have mostly been collected at the wireless client and AP. We extract these metrics from driver logs and derived statistic from them. Additionally we setup a wireless sniffer to get wireless captures. We use the wireless captures to validate the values we get from the logs at wireless client and AP. In the case of active metrics, we collect them from a wired client. The wired client works as the device in which we target to deploy our tool. From the wired client we trigger the active probing tool to collect RTT and throughput. The RTT measurements are collected using a custom ping-like tool. The active throughput is collected with \emph{iPerf}. In section \ref{Wireless Bottleneck Detector} we share instrumentation details on the tools we work with to obtain the metric we work with.
